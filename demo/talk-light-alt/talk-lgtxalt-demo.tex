\documentclass[10pt,compress,
%              draft,
               xcolor={dvipsnames,table},
               hyperref={breaklinks}
              ]{beamer}
%%%%%
% Fonts, input, language and other nice things
% http://tex.stackexchange.com/q/44694/33413
% http://tex.stackexchange.com/q/664/33413
% 
\usepackage[T1]{fontenc} 
\usepackage[utf8]{inputenc}
\usepackage[english]{babel}
% Non-italised greek letters
\usepackage{textgreek}
% Roman numerals
\usepackage{romanbar}                           %Römische Zahlen mit Balken.
\usepackage{lmodern}

%%%%%
% Science related packages
%
%%%
% Mathematics
%
\usepackage{amsmath}
\usepackage{amssymb}
%%%
% Units
%
% Easy typesetting of SI conform units
\usepackage{siunitx}
\DeclareSIUnit  % Use non-SI units
  \calorie{cal} % calorie, i.e. kcal/mol
\DeclareSIUnit  
  \electron{e}  % electron, i.e. eV
%%%
% Chemistry
%
% Easy typesetting of formulae
\usepackage[version=4]{mhchem}
% Enumerating compounds
\usepackage{chemnum}
% Support for creating and using schemes
% http://tex.stackexchange.com/q/6478/33413
\usepackage{newfloat}
\DeclareFloatingEnvironment[
  fileext=los,
  listname=List of Schemes,
  name=Scheme,
  placement=tbhp,
%  within=section,
]{scheme}
%%%
% Misc.
%
% Manual set-up for captions
% this is predominantly handy if used for posters
\usepackage[font+=footnotesize, %small,
            labelfont+={footnotesize,bf},
            format=hang,
            singlelinecheck=yes, % no for poster
            justification=raggedright % justified for poster
           ]{caption}
% provides captions comand without using environments
\usepackage{capt-of}
% Nice tables
\usepackage{booktabs}

%%%%%
% Graphic settings
% http://tex.stackexchange.com/q/23075/33413
% http://tex.stackexchange.com/q/139401/33413
% 
\usepackage{graphicx}
\graphicspath{{./graphics/}}
% http://ctan.org/pkg/adjustbox
% Using the export key with adjustbox, this will load the graphicx package, 
% and allow you to use its keys as part of \includegraphics.
\usepackage[export]{adjustbox}                        
% Use animations within presentation
\usepackage{animate}
                                                      
%%%%%
% Beamer specific commands
%
% possible options are alttitle, light, dark (default), poster 
% specifying the latter uses its own templates, 
% so the former options have no effect
\usetheme[light,alttitle]{Mito}
% prevent counting the appendix as slides
\usepackage{appendixnumberbeamer}               
% Setting title, etc.
\title{Short and Catchy Title}
\subtitle{Long and boring subtitle with unnecessary explanations.}
\author[F. Bar]{Foo Bar}
\institute[Baz Inst.]{Baz Insititute}
\titlegraphic{\includegraphics[scale=2]{example-image-a}}
% example-image from https://www.ctan.org/pkg/mwe
% scale ensures that the image is too big and must be resized by the template
\date{the Internet, \today}
% The following command can be used to create a custom footline for 
% a poster or the alternative titlepage.
% This is optional, but it will be set with default content
% if the alternative titlepage is requested.
\posterfootline{\insertdate\hfill Very Important Symposium}
% The following command can be used to give extra content to the alt. titlepage.
% This is optional and won't be set if empty.
% the content is centered, centered (to the best of my knowledge)
\titlepageextra{\includegraphics[width=0.5\linewidth]{example-image}}

%%%%%
% Custom commands and hacks
%
% Only for this demonstration we like
\usepackage{blindtext}
%
% custom newcommands can go here, for example
\newcommand{\diff}{\mathrm{d}} % upright differential operator
%
%%%%% End preamble.

\begin{document}

%\maketitle
\frame[plain]{\titlepage}

\begin{frame}
  \frametitle{Table of Contents}
  \tableofcontents
\end{frame}

\section{Text and Lists}
\begin{frame}
\frametitle{\insertsection}
\framesubtitle{ordinary text}
\blindtext
\end{frame}

\begin{frame}
\frametitle{\insertsection}
\framesubtitle{itemised lists}
\blindlistlist[3]{itemize}[3]
\end{frame}

\begin{frame}
\frametitle{\insertsection}
\framesubtitle{enumerated lists}
\blindlistlist[3]{enumerate}[3]
\end{frame}

\section{Boxed Text and Lists}
\begin{frame}
\frametitle{\insertsection}
\framesubtitle{ordinary block}
\begin{block}{with ordinary text}
\blindtext
\end{block}
\end{frame}

\begin{frame}
\frametitle{\insertsection}
\framesubtitle{alerted block}
\begin{alertblock}{with ordinary text}
\blindtext
\end{alertblock}
\end{frame}

\begin{frame}
\frametitle{\insertsection}
\framesubtitle{block for examples}
\begin{exampleblock}{with ordinary text}
\blindtext
\end{exampleblock}
\end{frame}

\section{Figures and Tables}
\begin{frame}
\frametitle{\insertsection}
\centering
\includegraphics[width=0.5\textwidth]{example-image}
\captionof{figure}{This is an example image from the mwe package.}
\end{frame}

\begin{frame}
\frametitle{\insertsection}
\centering
\captionof{table}{This is an example table.}
\begin{tabular}{lrrr}
  \toprule
         & Column A & Column B & Column C \\
  \midrule
  Row 1  & A1       & B1       & C1       \\
  Row 2  & A2       & B2       & C2       \\
  Row 3  & A3       & B3       & C3       \\
  \bottomrule
\end{tabular}
\end{frame}

\section{Mathematics and Chemistry}
\begin{frame}
  \frametitle{\insertsection}
  \framesubtitle{Mathematics}
  Mathematics can be included in-line or as a display. 
  For the following examples we can substitute \(u=e^x\) and \(v=e^{-x}\) 
  to save us some writing pain.
  
  The fundamental tricks of algebra are adding zero \eqref{eq:addingzero} %
  or multiplying by one \eqref{eq:multiplyingone}.
  \begin{subequations}
    \begin{align}\label{eq:addingzero}
      \int \frac{\diff x}{1 + e^x} &= \int \frac{\diff u}{u (u + 1)}\\ 
        &= \int \diff u \frac{1 + u - u}{u (1 + u)} \\
        &= \int \frac{\diff u}{u} - \int \frac{\diff u}{1 + u}
        &&= \ldots
    \end{align}
  \end{subequations}
  \begin{subequations}
    \begin{align}\label{eq:multiplyingone}
      \int \frac{\diff x}{1 + e^x} &= \int \diff x \frac{e^{-x}}{1 + e^{-x}} \\
        &= -\int \frac{\diff v}{1 + v}
        &&= \ldots
    \end{align}
  \end{subequations}
\end{frame}

\begin{frame}
  \frametitle{\insertsection}
  \framesubtitle{Chemistry}
  Writing about chemistry can be a tricky, but additional packages can make it easier.
  It is similar to mathematics and can be quite easily incorporated into it.

  For example take the combustion of dihydrogen (\ce{H2}, \cmpd{dihydrogen})
  with dioxygen (\ce{O2}, \cmpd{dioxygen}) to form water (\ce{H2O}, \cmpd{water}).
  The reaction seems quite simple: \[\ce{H2 + O2 -> 2 H2O}.\]

  The mechanism is much more complex and even scheme~\ref{sch:mechanism} is incomplete.

  \begin{align*}
    \ce{H2 &<=>[\Delta T] H.}\\
    \ce{H. + O2 &-> HO. + O.}\\
    \ce{O. + H2 &-> HO. + H.}\\
    \ce{HO. + H. &-> H2O}
  \end{align*}
  \captionof{scheme}{Incomplete reaction mechanism of the combustion of \cmpd{dihydrogen} and \cmpd{dioxygen}.}
  \label{sch:mechanism}

\end{frame}

\section{Multiple columns}
\begin{frame}[t]
\frametitle{\insertsection}
\framesubtitle{symmetric columns}
\begin{columns}[T]
  \column{0.49\textwidth}
    \begin{block}{For example}
      Hello, here is some text without a
      meaning. This text should show
      what a printed text will look like at
      this place. If you read this text,
      you will get no information.
      Really? Is there no information? Is
      there a difference between this text
      and some nonsense?
    \end{block}
  \hfill
  \column{0.49\textwidth}
    \blindlistlist[3]{enumerate}[3]
\end{columns}
\end{frame}

\begin{frame}
\frametitle{\insertsection}
\framesubtitle{asymmetric columns}
\begin{columns}
  \begin{column}{0.59\textwidth}
    \begin{block}{For example}
      Hello, here is some text without a
      meaning. This text should show
      what a printed text will look like at
      this place. If you read this text,
      you will get no information.
      Really? Is there no information? Is
      there a difference between this text
      and some nonsense?
    \end{block}
  \end{column}\hfill
  \begin{column}{0.39\textwidth}
    \includegraphics[width=\linewidth]{example-image}%-1x1}
    %\includegraphics[width=\linewidth]{example-image-golden-upright}
  \end{column}
\end{columns}
\end{frame}

\section{Animation and Highlighting}
\begin{frame}[t]
\frametitle{\insertsection}
\begin{columns}[T]
  \column{0.49\textwidth}
  \begin{block}{For example}
      This block comes first.

      \alert<1>{And this is highlighted}

      And the \emph{usual} blindtext:

      Hello, here is some text without a
      meaning. This text should show
      what a printed text will look like at
      this place. If you read this text,
      you will get no information.
      Really? Is there no information? Is
      there a difference between this text
      and some nonsense?
    \end{block}
    \pause
  \column{0.49\textwidth}
    After a break it is revealed, that 
    \begin{itemize}[<+- | alert@+>]
      \item this is important
      \item and this also
      \item and this needs to be considered.
    \end{itemize}
    \vspace{2ex}
    \pause
    \includegraphics[width=\linewidth]{example-image}
\end{columns}
\end{frame}

\section*{Endcard}

%\setbeamertemplate{background canvas}{\includegraphics[width=1.0\paperwidth]{theend}}
\begin{frame}[plain]
  \begin{centering}
    \begin{beamercolorbox}[sep=16pt,center]{part title} 
      \usebeamerfont{section title}{Thank you for your attention.}
    \end{beamercolorbox} 
  \end{centering}
\end{frame}

\appendix
\setbeamertemplate{background canvas}[default]
\section{\appendixname}

\bgroup
\setbeamercolor{background canvas}{bg=black}
\begin{frame}[plain]{}
\end{frame}
\egroup

\begin{frame}
  \frametitle{\insertsection}
  Backup slides can be included here.

  \blindtext
\end{frame}


\begin{frame}[t]
\frametitle{Catchy Slide Title}
\framesubtitle{Boring subtitle}
\begin{columns}[T]
  \column{0.49\textwidth}
    \begin{block}{Normal Block}
      Sampletext
    \end{block}
    \vspace{1ex}
    \begin{alertblock}{Alerted Block}
      Sampletext
    \end{alertblock}
    \vspace{1ex}
    \begin{exampleblock}{Example Block}
      Sampletext
    \end{exampleblock}
    \vspace{1ex}
    \includegraphics[width=0.4\textwidth]{example-image}
    \captionof{figure}{Example image from mwe.}
    
  \column{0.49\textwidth}
    \blindlistlist[2]{enumerate}[2]
    \vspace{1ex}
    \blindlistlist[2]{itemize}[2]
    \vspace{1ex}
    \captionof{table}{This is an example table.}
      \begin{tiny}
        \begin{tabular}{lrr}
          \toprule
                 & Column A & Column B \\
          \midrule
          Row 1  & A1       & B1       \\
          Row 2  & A2       & B2       \\
          \bottomrule
        \end{tabular}
      \end{tiny}
\end{columns}
\end{frame}

\end{document}
